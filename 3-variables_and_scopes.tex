\section{Variables and Scopes}

\begin{frame}[fragile]
    \frametitle{Why does this happen?}
    \begin{lstlisting}[language=JavaScript]
        function outer() {
            function inner() {
                x = 5
            }
            inner();
        }
        outer();
        console.log(x);  // ==> 5
    \end{lstlisting}
\end{frame}

\begin{frame}[fragile]
    \frametitle{C, C++, and Java Variables Are \textit{Block Scoped}}
    \begin{lstlisting}[language=Java]
    public void f() {
        for (int i = 0; i<10; i++) {
            System.out.println(i);
        }
        // Cannot use `i` here.
    }
    \end{lstlisting}
\end{frame}

\begin{frame}[fragile]
    \frametitle{JavaScript Variables are \textbf{Not} Block Scoped.\\
                They are function scoped.}
    \begin{lstlisting}[language=JavaScript]
    function f() {
        for (var i = 0; i<10; i++) {
            console.log(i);
        }
        // Can use `i` here.
        console.log(i)  // ==> 10
    }\end{lstlisting}
\end{frame}

\begin{frame}[fragile]
    \frametitle{JavaScript Variable Declarations are \textit{Hoisted}}
    \begin{lstlisting}[language=JavaScript]
    function f() {
        // `i` is declared but uninitialized.
        console.log(i)  // ==> undefined
        var i = 42
        console.log(i)  // ==> 42
    }
    \end{lstlisting}
\end{frame}

\begin{frame}[fragile]
    \frametitle{Variable Lookups in a Nested Function?}
    \begin{lstlisting}[language=JavaScript]
    var x = new Object();
    function outer() {
        function inner() {
            return x;
        }
        return inner;
    }
    x === outer()();  // ==> true
    \end{lstlisting}

    The \texttt{x} variable which is read is the ``nearest'' variable with this
    name in the scope chain. The search goes: \texttt{inner}, \texttt{outer},
    then finally \texttt{window} (the global object).
\end{frame}

\begin{frame}[fragile]
    \frametitle{Variable Assignments in a Nested Function?}
    \begin{lstlisting}[language=JavaScript]
        function outer() {
            function inner() {
                x = 5
            }
            inner();
        }
        outer();
        console.log(x);  // ==> 5
    \end{lstlisting}

    If no \texttt{x} variable is found in searching the scope chaing, then this
    implicitly declares a variable \texttt{x} as a property of the global object
    and initializes it.\footnotemark

    \footnotetext[1]{As of ES5, one can prevent this foolishness by using strict
                     mode.}
\end{frame}

\begin{frame}[fragile]
    \frametitle{How Is the Scope Chain Determined?}
    For some user-defined function \texttt{f}, the parent scope of \texttt{f} is
    the scope in which \texttt{f} was instantiated.
    \begin{lstlisting}[language=JavaScript]
        function wrap(wrapped) {
            function wrapper() {
                return wrapped;
            }
            return wrapper;
        }
        var a = {a: "a"};
        var b = {b: "b"};
        var a_wrapper = wrap(a);
        var b_wrapper = wrap(b);
        a_wrapper() === a;  // ==> true
        b_wrapper() === b;  // ==> true
    \end{lstlisting}
\end{frame}

\begin{frame}[fragile]
    \frametitle{Our Surprising Example (Revisited)}
    This example illustrates how the global object is at the intersection of
    these three different language features: variable, properties, and
    invocation contexts.

    \begin{lstlisting}[language=JavaScript]
    function f() {
        x = new Object();
    }
    f()
    window.x === x && x === this.x  // true!
    \end{lstlisting}

    We cannot normally use \textit{property accesses} on our scopes: we don't
    have expressions which evaluate to a scope. But there's one exception: the
    global object.
\end{frame}
